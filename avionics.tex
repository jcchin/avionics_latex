\documentclass[heading.tex]{subfiles} 
\begin{document}

% \tableofcontents

\section{Introduction}
\subsection{Motivation}
This research was initiated as risk reduction for the NASA
Gondola for High Altitude Planetary Science (GHAPS) mission.
High altitude balloon missions carrying expensive science payloads
across harsh, difficult-to-access locations could greatly benefit from
improved descent and landing systems with the ability to mitigate landing
loads and improve general recoverability.

As a first step, the Rocket University development program developed a series
of payloads of increasing capability to monitor and evaluate various recovery
systems. Comprised mainly of consumer-of-the-shelf (COTS) electronics and
open-source software, this paper offers a few turn-key solutions for highly
cost-constrained science missions. Rather than re-inventing the wheel, these
solutions leverage open-source microcontroller platforms with large ecosystems
that offer both flexibility and simplicity.


\subsection{Balloon Sub-System Basics}

A vast multitude of HAB related resources exist online, therefore many of the
ancillirary subsystems are only breifly outlined here. With many amatuer missions
focused on videography, this work addresses data aquisition and power management;
two disciplines that scale to any size balloon mission.

\subsubsection{Power}

Desirable battery characterstics include:
Power Density: Amp-hours per unit of mass.
Temperature: Must be capable of handling -70 \deg C temperatures without
significant voltage drop.
Discharge Characteristics: Capable of handling sufficient current draw without
significant voltage drop.

Lithium Sulfer Dioxide (LiSO2) non-rechargeable batteries satisfy all of these
requirements and are readily available from SAFT batteries.
This particular battery type is rated for many space and medical applications,
and also happen to be the primary battery type chosen by the Balloon Project Office.


\subsubsection{Telemetry}

The ability to recieve live wirelessly transmitted data from a payload is
essential for the recoverability of any balloon launch. FindMeSpot is the easiest
COTS device, requiring no configuration or operator licenses. Communication via
cell phone devices also offers slightly more flexibility, but additional configuration.
Both of these methods are limited to 40,000ft altitude celings. To maintain constant
communication up to 100,000+ feet requires higher powered radios. The most
accessible amatuer radios require an operator with an amatuer HAM license.
These licenses require taking an exam (technical difficulty on par with a
driving exam) from a local HAM chapter. Many free online resources and study
guides provide an easy way to master the finite pool of possible written quiz
questions in a few weeks time. After passing the written test, the basic
technician license is granted by the FCC within a few weeks. Additional licenses
can be obtained by taking increasingly difficult written exams, allowing the
operation of higher powered radios over a larger range of frequencies.

The most popular balloon radio protocol, Automatic Packet Reporting System (APRS)
provides a very succinct transmission including the balloons GPS lat, lon, heading,
and operator call-sign. By commiting to this protocol, the operator is able to
leverage a large amount of pre-existing HAM radio infrastrucutre. Rather than
relying exclusively on personal ground station equipment, many other amatuer
stations are constantly listening for the APRS frequency and protocol. If the
signal reaches any of these stations, the signal will be repeated and re-broadcasted
to neighboring stations. Eventually the signal will reach an internet ``gateway''
where it can published to various web applications. This allows an APRS operator
to simply broadcast signals, without the direct need for any reciever equipment.
As long as the signal reaches neighboring HAM stations, the call-sign and information
will be available in real-time from any internet connected device. For more direct
reception, many affordable handheld devices [list here] are available that can
be tuned to filter APRS messages from specfic call-signs.

If the payload requires more advanced telemetry, leveraging other frequencies
and protocols may be more suitable. If a constant transmission
(possibly bi-directional) communication link or higher bandwidth is needed,
remote control UAV radio components may be a better option. With balloon
operations requiring extremely long-distance communication, the most ideal
frequencies reside in the UHF range. As a rule of thumb, transmission bandwith
increases and range decreases with increasing frequency. 900Mhz is ideal for
basic telemetry in the 100-200 kbps range up to 40 miles away. Radio modems
include the RFD900, and Xtend900. These specific models are identified for their
drop-in compatibility with existing hobby level UAV ecosystems that are outlined
in subsequent sections. They are only modestly export controlled, require only
a basic technician's license, and require the minimal amount of radio and 
programming experience.

\subsubsection{Balloon and Parachute}

Include images, and balloon providers,where to purchase, estimated helium cost,

\subsubsection{Gondola Design}

\<6lb Foam core design, with images/diagrams!

BFS gondola design

\subsubsection{Launch Operations}

Balloon: Inflation procedures, latex gloves, tarp, paddles, etc.

Pre-flight: estimating trajectory, launch supplies
temperature profile, lift calcs vs cost, 

FAA regs: NOTAM, etc

Developing FAA regulations (as of February 2014) limit small UAS (\<55lbs) to
line of site flight, \<500 ft altitude ceiling, daylight operation, operator certification,
and additional restrictions in various airspaces (such as 5 mile radii around airports)
without notifying air traffic control or further FAA exemption.
A separate, ``more felxible'', set of regulations are under considerations
for ``micro UAS'' class aircraft that fall under 4.4 lbs.
Although it's unclear how balloon systems overlap with UAS, the development
of guided descent and recovery systems certainly fall under both categories.
[Add Citations http://www.faa.gov/regulations_policies/rulemaking/media/021515_sUAS_Summary.pdf]

\section{Avionics for Multiple Cost Levels}

Microcontrollers and systems on a chip (SoC) have improved significantly in 
compute power, accessibility, and cost over the past few years.
Initially driven by hobby electronics, an extremely broad community of developers
have emerged, blurring the lines between amatuer and professional systems for
many applications. The new economies of scale, trickle-down smartphone
technology, Internet-of-Things (IoT) movement, and beginner friendly developer
tools have made high-performance sensing platforms accessible to anyone.

The choice of hardware outlined here is separated into various pricing teirs,
each optimized for rapid development time and extensibility.
%\begin{figure}[H]
%\centering
%\includegraphics[width=1.0\textwidth]{images/optimum_product_vs_time}
%\caption{Overarching Applied Research Challenge: Uncertainty in Optimum Product vs. Time \& Cost to Produce}
%\label{f:product_vs_time}
%\end{figure}
\subsection{Sub \$200 launches}

Micro-HABS

\subsection{Sub \$1,000 launches}

Arduino Based

\subsection{Sub \$2,000 launches}

Pixhawk Based

\section{Power Management and Distribution}

Anthony's work goes here.

\section{Visualizing Results}

\subsection{Processing}

Processing is a programming language and development environment designed
with visualization in mind. For this project, it was used to sync video
footage with visualized sensor data. The program embeds the video and syncs
the framerate with rows of sensor data from a CSV. At 30 frames per second,
it reads a new row from the CSV file every 6 frames, updating the display.

\subsection{Data Dissemination}

Google Fusion Tables, Github, etc

\section{Final Remarks}
The sensor packages outlined in this paper are inconsequential in weight and
compatible for payloads of many sizes up to thousands of pounds.
These systems were originally intended for passive monitoring systems, however the final avionics system outlined could be used to control an active descent system.

Payloads falling under the FAA exempt payload regulation offer exponentially
lower costs with vastly fewer restrictions. 


\end{document}

