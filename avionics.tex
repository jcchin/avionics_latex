\documentclass[heading.tex]{subfiles} 
\begin{document}

% \tableofcontents

\section{Introduction}
\subsection{Motivation}
This research was initiated as risk reduction for the NASA
Gondola for High Altitude Planetary Science (GHAPS) mission.
High altitude balloon (HAB) missions carrying expensive science payloads
across harsh, difficult-to-access locations could greatly benefit from
improved descent and landing systems with the ability to mitigate landing
loads and improve general recoverability.

As a first step, the Rocket University training program developed a series of
payloads of increasing capability to monitor and evaluate various recovery
systems. Comprised of consumer-of-the-shelf (COTS) electronics and open-source
software, this paper offers a few turn-key solutions for highly cost-constrained
science missions. Rather than re-inventing the wheel, these solutions leverage
open-source microcontroller platforms with large ecosystems that offer both
flexibility and simplicity.


\subsection{Balloon Sub-Systems}

\subsubsection{Balloon}
\subsubsection{Parachute}
\subsubsection{Telemetry}
\subsubsection{Power}

\section{Avionics for Multiple Cost Levels}

%\begin{figure}[H]
%\centering
%\includegraphics[width=1.0\textwidth]{images/optimum_product_vs_time}
%\caption{Overarching Applied Research Challenge: Uncertainty in Optimum Product vs. Time \& Cost to Produce}
%\label{f:product_vs_time}
%\end{figure}
\subsection{Sub \$500 launches}

\subsection{Sub \$1,000 launches}

\subsection{Sub \$2,000 launches}

\section{Visualizing Results}


\subsection{Processing}

\subsection{Google Fusion Tables}


\section{Final Remarks}
The sensor packages outlined in this paper are inconsequential in weight and
compatible for payloads of many sizes up to thousands of pounds. These systems
were originally intended for passive monitoring systems, however the final
avionics system outlined could be used to control an active descent system.

Payloads falling under the FAA exempt payload regulation offer exponentially
lower costs with vastly fewer restrictions. 


\end{document}

