\documentclass[heading.tex]{subfiles} 
\begin{document}

% \tableofcontents

\section{Introduction}
\subsection{Motivation}
This research was initiated as risk reduction for the NASA
Gondola for High Altitude Planetary Science (GHAPS) mission.
High altitude balloon missions carrying expensive science payloads
across harsh, difficult-to-access locations could greatly benefit from
improved descent and landing systems with the ability to mitigate landing
loads and improve general recoverability.

As a first step, the Rocket University development program developed a series
of payloads of increasing capability to monitor and evaluate various recovery
systems. Comprised mainly of consumer-of-the-shelf (COTS) electronics and
open-source software, this paper offers a few turn-key solutions for highly
cost-constrained science missions. Rather than re-inventing the wheel, these
solutions leverage open-source microcontroller platforms with large ecosystems
that offer both flexibility and simplicity.


\subsection{Balloon Sub-System Basics}

A vast multitude of HAB related resources exist online, therefore many of the
ancillirary subsystems are only breifly outlined here. With many amatuer missions
focused on videography, this work addresses data aquisition and power management;
two disciplines that scale to any size balloon mission.

\subsubsection{Power}

LiSO2

\subsubsection{Telemetry}

RFD900, APRS

\subsubsection{Balloon and Parachute}

Include images, and balloon providers

\subsubsection{Gondola Design}

<6lb Foam core design, BFS gondola design

\subsubsection{Launch Operations}

Inflating the balloon, estimating trajectory, launch supplies
temperature profile, lift calcs vs cost, 

\section{Avionics for Multiple Cost Levels}

Microcontrollers and systems on a chip (SoC) have improved significantly in 
compute power, accessibility, and cost over the past few years.
Initially driven by hobby electronics, an extremely broad community of developers
have emerged, blurring the lines between amatuer and professional systems for
many applications. The new economies of scale, trickle-down smartphone
technology, Internet-of-Things (IoT) movement, and beginner friendly developer
tools have made high-performance sensing platforms accessible to anyone.

The choice of hardware outlined here is separated into various pricing teirs,
each optimized for rapid development time and extensibility.
%\begin{figure}[H]
%\centering
%\includegraphics[width=1.0\textwidth]{images/optimum_product_vs_time}
%\caption{Overarching Applied Research Challenge: Uncertainty in Optimum Product vs. Time \& Cost to Produce}
%\label{f:product_vs_time}
%\end{figure}
\subsection{Sub \$200 launches}

Micro-HABS

\subsection{Sub \$1,000 launches}

Arduino Based

\subsection{Sub \$2,000 launches}

Pixhawk Based

\section{Power Management and Distribution}

Anthony's work goes here.

\section{Visualizing Results}

\subsection{Processing}

Processing is a programming language and development environment designed
with visualization in mind. For this project, it was used to sync video
footage with visualized sensor data. The program embeds the video and syncs
the framerate with rows of sensor data from a CSV. At 30 frames per second,
it reads a new row from the CSV file every 6 frames, updating the display.

\subsection{Data Dissemination}

Google Fusion Tables, Github, etc

\section{Final Remarks}
The sensor packages outlined in this paper are inconsequential in weight and
compatible for payloads of many sizes up to thousands of pounds.
These systems were originally intended for passive monitoring systems, however the final avionics system outlined could be used to control an active descent system.

Payloads falling under the FAA exempt payload regulation offer exponentially
lower costs with vastly fewer restrictions. 


\end{document}

